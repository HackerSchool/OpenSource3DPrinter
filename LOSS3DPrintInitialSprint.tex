\documentclass[a4paper,12pt]{article}
\usepackage{graphicx}
\usepackage[a4paper, total={6in, 9in}]{geometry}
\usepackage[T1]{fontenc}
\usepackage[utf8]{inputenc}
\usepackage{graphicx}
\usepackage{tikz}
\usepackage{float}
\usepackage{mathtools}
\usepackage{fancyhdr}
\usepackage{caption}
\usepackage{textgreek}
\usepackage{yfonts}
\graphicspath{/home/}
\pagestyle{fancy}
\date{Março 2022}
\title{ \\ \large {Initial Sprint Report}}
\author{
\\ João Barreiros C. Rodrigues
\\}

\begin{document}
	\pagenumbering{gobble}
	\begin{titlepage} % Suppresses displaying the page number on the title page and the subsequent page counts as page 1
        \newcommand{\HRule}{\rule{\linewidth}{0.5mm}} % Defines a new command for horizontal lines, change thickness here
        \center % Centre everything on the page
        \textsc{\LARGE Instituto Superior Técnico}\\[1.5cm] % Main heading such as the name of your university/college
	\textsc{\Large HackerSchool}\\[0.25cm]
        \HRule\\[0.4cm]
        {\LARGE\bfseries Initial Sprint Report}\\[0.4cm] % Title of your document
	{\huge\bfseries Free Libre Open Source 3D Printer}\\[0.4cm] % Title of your document
        \HRule\\[1.5cm]\
	Manuel \textsc{Soares}, LEEC aka \\textsc{Nuclear Monk},\
        João \textsc{Barreiros C. Rodrigues}, LEEC aka \textsc{Ex-Machina},\\
	Nuno \textsc{Tribolet Abreu}, LEEC aka \textsc{Wone_Bone},\\
        \vfill\vfill\vfill % Position the date 3/4 down the remaining page
        {\large March 2022} % Date, change the \today to a set date if you want to be precise
        \vfill % Push the date up 1/4 of the remaining page
\end{titlepage}
	\pagenumbering{arabic}
	\newpage
		\tableofcontents
	\section{Introduction and Motivation}
	%Introduction to the project
	
	\section{Theoretical references for the Project}
	%Helpful links, books and other sources
	
	\section{Project specifics}
		
		\subsection{Project insertion on the Hacker subject}
		
		%Areas of interest, depeen your introduction and motivation
		\subsection{Task planning and management}
			\subsection{Task planning, partial and total ETC}
		%Splice your project into easily digestible tasks
		%You can use a tabular enviroment for this
		
		\begin{table}[h]
			\begin{tabular}{lllll}
				\hline
				Task \# & Task Description & Task Dependencies & Materials Needed & ETC \\ \hline
        			0 & Inventory the printer's corpse, take note of missing pieces & N/A & N/A & 1 day \\
        			1 & Solder &                   &                  &     \\
				2 & &                   &                  &     \\ \hline
			\end{tabular}
		\end{table}
		\textbf{Total ETC} : %Antevise the ETC/ETA. Might be helpful to do it after task plannning.
		\subsubsection{Materials}
		%Gather all the materials mentioned previously
		%You can use a tabular enviroment for this
		\begin{table}[h]
			\begin{tabular}{llll}
			\hline
			Material Description & In Storage? & Cost & References \\ \hline
                     	&             &      &            \\
                     	&             &      &            \\
                     	&             &      &            \\ \hline
			\end{tabular}
		\end{table}


\end{document}

